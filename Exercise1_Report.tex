\documentclass[11pt]{article}
\usepackage[utf8]{inputenc}
\usepackage{geometry}
\usepackage{booktabs}
\usepackage{listings}
\usepackage{xcolor}
\usepackage{amsmath}
\usepackage{hyperref}

\geometry{margin=1in}

\lstset{
    basicstyle=\ttfamily\footnotesize,
    breaklines=true,
    frame=single,
    showstringspaces=false,
    columns=flexible
}

\title{Exercise 1: Prompting, Debugging, and Innovation\\for Code Generation with LLMs}
\author{Anh Tran \\ CS520 - Programming Languages}
\date{\today}

\begin{document}

\maketitle

\section{Part 1: Prompt Design and Code Generation}

\subsection{LLM Families}

Two distinct model families were selected:

\begin{itemize}
    \item \textbf{Gemini 2.5 Flash} (Google DeepMind) - API via \texttt{models/gemini-2.5-flash}
    \item \textbf{Mistral Large} (Mistral AI) - API via \texttt{mistral-large-latest}
\end{itemize}

\subsection{Problems Selected}

Ten problems from HumanEval+ dataset \cite{humanevalplus}:

\begin{table}[h]
\centering
\small
\begin{tabular}{lll}
\toprule
\textbf{ID} & \textbf{Function} & \textbf{Type} \\
\midrule
HumanEval/0 & has\_close\_elements & List comparison \\
HumanEval/1 & separate\_paren\_groups & String parsing \\
HumanEval/10 & make\_palindrome & String manipulation \\
HumanEval/12 & longest & List operation \\
HumanEval/17 & parse\_music & String to list \\
HumanEval/25 & factorize & Number theory \\
HumanEval/31 & is\_prime & Number theory \\
HumanEval/54 & same\_chars & String comparison \\
HumanEval/61 & correct\_bracketing & Stack algorithm \\
HumanEval/108 & count\_nums & Number processing \\
\bottomrule
\end{tabular}
\caption{Selected problems}
\end{table}

\subsection{Prompting Strategies}

\textbf{Chain-of-Thought (CoT):}
\begin{lstlisting}
Task: Implement the target function in Python.

Chain-of-Thought:
1) Restate the contract and constraints.
2) Reason about edge cases and examples.
3) Identify a correct and simple algorithm.
4) Implement the function.

Output only the function implementation.

Problem: [description]
Function name: [name]
\end{lstlisting}

\textbf{Stepwise Chain-of-Thought (SCoT):}
\begin{lstlisting}
Task: Implement the target function as specified.

Stepwise Chain-of-Thought Plan:
1) Restate the function contract precisely.
2) Identify edge cases and constraints.
3) Outline a step-by-step algorithm.
4) Implement the function in Python.
5) Double-check correctness on edge cases.

Output: Only the function implementation.

Problem: [description]
Function name: [name]
\end{lstlisting}

\subsection{Experimental Setup}

\begin{itemize}
    \item Generated: 40 solutions (10 problems $\times$ 2 families $\times$ 2 strategies)
    \item API Parameters: Temperature 0.7, Max tokens 2000
    \item Evaluation: HumanEval+ test suites with 10-second timeout
    \item Metric: pass@k = $1 - \frac{\binom{n-c}{k}}{\binom{n}{k}}$
\end{itemize}

\subsection{Results}

\begin{table}[h]
\centering
\footnotesize
\begin{tabular}{llcc}
\toprule
\textbf{Problem} & \textbf{Family/Strategy} & \textbf{Result} & \textbf{pass@1} \\
\midrule
\multirow{4}{*}{humaneval\_0} & gemini/cot & 1/1 & 1.000 \\
& gemini/scot & 1/1 & 1.000 \\
& mistral/cot & 1/1 & 1.000 \\
& mistral/scot & 1/1 & 1.000 \\
\midrule
\multicolumn{4}{c}{... (humaneval\_1, 10, 12, 17, 31, 61 all passed) ...} \\
\midrule
\multirow{4}{*}{\textbf{humaneval\_25}} & gemini/cot & 1/1 & 1.000 \\
& gemini/scot & 1/1 & 1.000 \\
& mistral/cot & \textbf{0/1} & \textbf{0.000} \\
& mistral/scot & \textbf{0/1} & \textbf{0.000} \\
\midrule
\multirow{4}{*}{\textbf{humaneval\_54}} & gemini/cot & 1/1 & 1.000 \\
& gemini/scot & 1/1 & 1.000 \\
& mistral/cot & \textbf{0/1} & \textbf{0.000} \\
& mistral/scot & \textbf{0/1} & \textbf{0.000} \\
\midrule
\multirow{4}{*}{\textbf{humaneval\_108}} & gemini/cot & \textbf{0/1} & \textbf{0.000} \\
& gemini/scot & 1/1 & 1.000 \\
& mistral/cot & \textbf{0/1} & \textbf{0.000} \\
& mistral/scot & \textbf{0/1} & \textbf{0.000} \\
\bottomrule
\end{tabular}
\caption{Results (failures in bold)}
\end{table}

\begin{table}[h]
\centering
\begin{tabular}{lcccc}
\toprule
\textbf{Family} & \textbf{Strategy} & \textbf{Pass} & \textbf{Fail} & \textbf{Rate} \\
\midrule
Gemini & CoT & 9 & 1 & 90\% \\
Gemini & SCoT & 10 & 0 & \textbf{100\%} \\
Mistral & CoT & 7 & 3 & 70\% \\
Mistral & SCoT & 7 & 3 & 70\% \\
\midrule
\textbf{Overall} & & \textbf{33} & \textbf{7} & \textbf{82.5\%} \\
\bottomrule
\end{tabular}
\caption{Summary by family and strategy}
\end{table}

\subsection{Analysis}

\textbf{Key Findings:}
\begin{enumerate}
    \item Gemini (95\%) outperforms Mistral (70\%)
    \item SCoT improves Gemini to 100\% but doesn't help Mistral
    \item 7 failures identified across 3 problems
\end{enumerate}

\textbf{Failure Types:}
\begin{itemize}
    \item humaneval\_25/mistral: Timeout (inefficient algorithm)
    \item humaneval\_54/mistral: Logic error
    \item humaneval\_108: Edge case handling (negative numbers)
\end{itemize}

\newpage
\section{Part 2: Debugging and Iterative Improvement}

\subsection{Failure Case 1: [Problem Name]}

\textbf{Failed Solution:}
\begin{itemize}
    \item Problem: [ID]
    \item Family: [gemini/mistral]
    \item Strategy: [cot/scot]
    \item Error Type: [description]
\end{itemize}

\textbf{Original Code:}
\begin{lstlisting}
[paste failing code here]
\end{lstlisting}

\textbf{Self-Repair Prompt Used:}
\begin{lstlisting}
[paste self-repair prompt here]
\end{lstlisting}

\textbf{Fixed Code:}
\begin{lstlisting}
[paste fixed code here]
\end{lstlisting}

\textbf{Analysis:}
\begin{itemize}
    \item Why it failed: [explanation]
    \item How fix works: [explanation]
    \item Result: [pass@1 after repair]
\end{itemize}

\subsection{Failure Case 2: [Problem Name]}

[Same structure as above]

\subsection{Cross-Family Comparison}

[Discuss how different families handled the same failures]

\newpage
\section{Part 3: Innovation - Proposed Strategy}

\subsection{Novel Strategy Description}

\textbf{Strategy Name:} [Your innovation name]

\textbf{Motivation:} [Why you designed this]

\textbf{Workflow:} [Step-by-step description]

\subsection{Prompt Template}

\begin{lstlisting}
[paste your innovation prompt here]
\end{lstlisting}

\subsection{Implementation}

Generated solutions for both families across [X] problems.

\subsection{Results}

\begin{table}[h]
\centering
\begin{tabular}{lccc}
\toprule
\textbf{Strategy} & \textbf{Gemini} & \textbf{Mistral} & \textbf{Overall} \\
\midrule
CoT & 90\% & 70\% & 80\% \\
SCoT & 100\% & 70\% & 85\% \\
\textbf{Innovation} & \textbf{[X]\%} & \textbf{[Y]\%} & \textbf{[Z]\%} \\
\bottomrule
\end{tabular}
\caption{Innovation strategy comparison}
\end{table}

\subsection{Analysis}

\textbf{Did it improve results?} [Yes/No and why]

\textbf{Differences across families:} [How Gemini vs Mistral responded]

\textbf{Lessons learned:} [What worked, what didn't]

\section{Overall Conclusion}

\textbf{Summary:}
\begin{itemize}
    \item Part 1: 82.5\% success (33/40), Gemini superior
    \item Part 2: [X] failures debugged successfully
    \item Part 3: [Innovation result]
\end{itemize}

\textbf{Key Insights:}
\begin{itemize}
    \item Structured prompting helps capable models
    \item Algorithmic knowledge varies between families
    \item [Your insight from innovation]
\end{itemize}

\section{Repository}

All code, prompts, tests, and results:\\
\url{https://github.com/[your-username]/exercise_1}

\begin{thebibliography}{9}
\bibitem{humanevalplus}
EvalPlus Team. HumanEval+ Dataset. 
\url{https://huggingface.co/datasets/evalplus/humanevalplus}, 2023.
\end{thebibliography}

\end{document}

