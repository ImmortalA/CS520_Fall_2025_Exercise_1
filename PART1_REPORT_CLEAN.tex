\documentclass[11pt]{article}
\usepackage[utf8]{inputenc}
\usepackage{geometry}
\usepackage{booktabs}
\usepackage{listings}
\usepackage{xcolor}
\usepackage{amsmath}
\geometry{margin=1in}

\lstset{
    basicstyle=\ttfamily\small,
    breaklines=true,
    frame=single,
    showstringspaces=false,
    columns=flexible
}

\usepackage{hyperref}


\title{Exercise 1}
\author{CS520 - Anh Tran}
\date{\today}

\begin{document}

\maketitle

\section{Part 1}

We selected two model families from different providers:

\begin{itemize}
    \item \textbf{Gemini 2.5 Flash} (Google DeepMind family)
    \begin{itemize}
        \item Model: \texttt{models/gemini-2.5-flash}
        \item Access Method: Google AI Studio API
        \item Provider: Google DeepMind
    \end{itemize}
    
    \item \textbf{Mistral Large} (Mistral AI family)
    \begin{itemize}
        \item Model: \texttt{mistral-large-latest}
        \item Access Method: Mistral La Plateforme API
        \item Provider: Mistral AI
    \end{itemize}
\end{itemize}

These two families represent distinct architectural approaches and training methodologies, ensuring diverse evaluation perspectives as required by the assignment.

\section{Problem Selection}

We selected 10 problems from the HumanEval+ dataset, chosen to span different difficulty levels and algorithm types:

\begin{table}[h]
\centering
\small
\begin{tabular}{llp{7cm}}
\toprule
\textbf{Problem ID} & \textbf{Function} & \textbf{Description} \\
\midrule
HumanEval/0 & has\_close\_elements & Check if any two numbers are closer than threshold \\
HumanEval/1 & separate\_paren\_groups & Parse nested parentheses groups \\
HumanEval/10 & make\_palindrome & Create palindrome from string \\
HumanEval/12 & longest & Find longest string in list \\
HumanEval/17 & parse\_music & Convert music notation to integers \\
HumanEval/25 & factorize & Return prime factors \\
HumanEval/31 & is\_prime & Check if number is prime \\
HumanEval/54 & same\_chars & Check if strings have same chars \\
HumanEval/61 & correct\_bracketing & Validate bracket balancing \\
HumanEval/108 & count\_nums & Count numbers with positive digit sum \\
\bottomrule
\end{tabular}
\caption{Selected problems from HumanEval+ dataset}
\label{tab:problems}
\end{table}

\textbf{Rationale:} These problems cover diverse tasks: list operations, string manipulation, number theory, and data structures.

\section{Prompting Strategies}

\subsection{Chain-of-Thought (CoT)}

\begin{lstlisting}[caption={CoT Prompt Template}]
Task: Implement the target function in Python.

Chain-of-Thought:
1) Restate the contract and constraints.
2) Reason about edge cases and examples.
3) Identify a correct and simple algorithm.
4) Implement the function.

Output only the function implementation.

Problem:
[problem description]

Function name: [function_name]

IMPORTANT: Output ONLY the function implementation code.
\end{lstlisting}

\subsection{Stepwise Chain-of-Thought (SCoT)}

\begin{lstlisting}[caption={SCoT Prompt Template}]
Task: Implement the target function as specified.

Stepwise Chain-of-Thought Plan:
1) Restate the function contract precisely.
2) Identify edge cases and constraints.
3) Outline a step-by-step algorithm.
4) Implement the function in Python.
5) Double-check correctness on edge cases.

Output: Only the function implementation.

Problem:
[problem description]

Function name: [function_name]

IMPORTANT: Output ONLY the function implementation code.
\end{lstlisting}

\section{Experimental Setup}

\subsection{Generation Process}

For each of the 10 problems, we generated solutions using:
\begin{itemize}
    \item 2 model families (Gemini, Mistral)
    \item 2 prompting strategies (CoT, SCoT)
    \item Total: 40 generated solutions
\end{itemize}

All generations were performed via API calls with:
\begin{itemize}
    \item Temperature: 0.7
    \item Max tokens: 2000
    \item System message: ``You are an expert Python programmer. Output only code.''
\end{itemize}

\subsection{Evaluation Methodology}

Solutions were evaluated using HumanEval+ comprehensive test suites. The pass@k metric was computed using:

\[ \text{pass}@k = 1 - \frac{\binom{n-c}{k}}{\binom{n}{k}} \]

where $n$ is the number of samples and $c$ is the number of correct solutions.

A 10-second timeout was implemented to catch inefficient implementations.

\section{Results}

\subsection{Complete Results Table}

\begin{table}[h]
\centering
\footnotesize
\begin{tabular}{llcccc}
\toprule
\textbf{Problem} & \textbf{Family} & \textbf{Strategy} & \textbf{Correct} & \textbf{pass@1} \\
\midrule
humaneval\_0 & gemini & cot & 1/1 & 1.000 \\
 & gemini & scot & 1/1 & 1.000 \\
 & mistral & cot & 1/1 & 1.000 \\
 & mistral & scot & 1/1 & 1.000 \\
\midrule
humaneval\_1 & gemini & cot & 1/1 & 1.000 \\
 & gemini & scot & 1/1 & 1.000 \\
 & mistral & cot & 1/1 & 1.000 \\
 & mistral & scot & 1/1 & 1.000 \\
\midrule
humaneval\_10 & gemini & cot & 1/1 & 1.000 \\
 & gemini & scot & 1/1 & 1.000 \\
 & mistral & cot & 1/1 & 1.000 \\
 & mistral & scot & 1/1 & 1.000 \\
\midrule
humaneval\_12 & gemini & cot & 1/1 & 1.000 \\
 & gemini & scot & 1/1 & 1.000 \\
 & mistral & cot & 1/1 & 1.000 \\
 & mistral & scot & 1/1 & 1.000 \\
\midrule
humaneval\_17 & gemini & cot & 1/1 & 1.000 \\
 & gemini & scot & 1/1 & 1.000 \\
 & mistral & cot & 1/1 & 1.000 \\
 & mistral & scot & 1/1 & 1.000 \\
\midrule
humaneval\_31 & gemini & cot & 1/1 & 1.000 \\
 & gemini & scot & 1/1 & 1.000 \\
 & mistral & cot & 1/1 & 1.000 \\
 & mistral & scot & 1/1 & 1.000 \\
\midrule
humaneval\_61 & gemini & cot & 1/1 & 1.000 \\
 & gemini & scot & 1/1 & 1.000 \\
 & mistral & cot & 1/1 & 1.000 \\
 & mistral & scot & 1/1 & 1.000 \\
\midrule
\textbf{humaneval\_25} & gemini & cot & 1/1 & 1.000 \\
 & gemini & scot & 1/1 & 1.000 \\
 & mistral & cot & 0/1 & 0.000 \\
 & mistral & scot & 0/1 & 0.000 \\
\midrule
\textbf{humaneval\_54} & gemini & cot & 1/1 & 1.000 \\
 & gemini & scot & 1/1 & 1.000 \\
 & mistral & cot & 0/1 & 0.000 \\
 & mistral & scot & 0/1 & 0.000 \\
\midrule
\textbf{humaneval\_108} & gemini & cot & 0/1 & 0.000 \\
 & gemini & scot & 1/1 & 1.000 \\
 & mistral & cot & 0/1 & 0.000 \\
 & mistral & scot & 0/1 & 0.000 \\
\bottomrule
\end{tabular}
\caption{Complete evaluation results (failures in bold)}
\label{tab:results}
\end{table}

\subsection{Performance Summary}

\begin{table}[h]
\centering
\begin{tabular}{lcccc}
\toprule
\textbf{Family} & \textbf{Strategy} & \textbf{Passed} & \textbf{Failed} & \textbf{Rate} \\
\midrule
Gemini & CoT & 9 & 1 & 90.0\% \\
Gemini & SCoT & 10 & 0 & 100.0\% \\
Mistral & CoT & 7 & 3 & 70.0\% \\
Mistral & SCoT & 7 & 3 & 70.0\% \\
\midrule
\textbf{Overall} & & \textbf{33} & \textbf{7} & \textbf{82.5\%} \\
\bottomrule
\end{tabular}
\caption{Performance summary by family and strategy}
\label{tab:summary}
\end{table}

\subsection{Key Observations}

\begin{enumerate}
    \item \textbf{Gemini Outperforms Mistral}: Gemini achieved 95\% vs 70\% success rate.
    
    \item \textbf{SCoT Improves Gemini}: SCoT achieved 100\% for Gemini vs 90\% for CoT.
    
    \item \textbf{Mistral Consistent Across Strategies}: Both CoT and SCoT achieved 70\% for Mistral.
    
    \item \textbf{Problem-Specific Failures}: 
    \begin{itemize}
        \item humaneval\_25 (factorize): Mistral timeout on large primes
        \item humaneval\_54 (same\_chars): Mistral logic error
        \item humaneval\_108 (count\_nums): Complex negative number handling
    \end{itemize}
\end{enumerate}

\section{Analysis}

\subsection{Perfect Performance (7/10 problems)}

Both families achieved 100\% on: humaneval\_0, humaneval\_1, humaneval\_10, humaneval\_12, humaneval\_17, humaneval\_31, humaneval\_61.

\subsection{Failure Analysis}

\textbf{humaneval\_25 (factorize)}: Mistral generated inefficient trial division without square root optimization, causing timeouts.

\textbf{humaneval\_54 (same\_chars)}: Mistral failed on character comparison logic.

\textbf{humaneval\_108 (count\_nums)}: Edge case handling for negative numbers. SCoT helped Gemini but not Mistral.

\section{Conclusion}

Part 1 Results:
\begin{itemize}
    \item Gemini 2.5 Flash: 95\% success (SCoT: 100\%)
    \item Mistral Large: 70\% success
    \item Overall: 82.5\% (33/40)
    \item 7 failures identified for Part 2 debugging
\end{itemize}

SCoT prompting significantly improves Gemini performance but not Mistral, suggesting more capable models benefit more from structured prompting.

\section{Repository}

Code and results: \url{https://github.com/[your-username]/exercise_1}

\begin{thebibliography}{9}
\bibitem{humanevalplus}
EvalPlus Team. HumanEval+ Dataset. \url{https://huggingface.co/datasets/evalplus/humanevalplus}, 2023.
\end{thebibliography}

\end{document}

